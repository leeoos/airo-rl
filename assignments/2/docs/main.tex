\documentclass[10pt,a4paper]{article}

% Standard required packages
\usepackage[utf8]{inputenc}
\usepackage{amsmath,bm,mathtools,amssymb} % math packages
\usepackage{listings}
\usepackage{graphicx}
\usepackage{caption}
\usepackage{subcaption}
\usepackage{hyperref}
\hypersetup{
    colorlinks,
    citecolor=black,
    filecolor=black,
    linkcolor=black,
    urlcolor=black
}

\begin{document}

\begin{titlepage}
    \begin{center}
        \vspace*{1cm}
        \Huge\textbf{Assignment 2}\\
        \vspace{1.5cm}
        \Large Author:
        \textbf{Leonardo Colosi 1799057}\\
        \vspace{0.5cm}
        \Large Contributors: \textbf{Bruno Francesco Nocera 1863075, Silverio Manganaro 1817504, Simone Tozzi, 1615930, Paolo Renzi 1887793, Jacopo Tedeschi 1882789, Amine Ahardane 2050689.}
        \vfill
        \includegraphics[width=0.7\textwidth]{images/sapienza_logo.png}
        \vfill
        \vspace{0.8cm}
        \Large \textit{MARR, RL}\\
        \today
    \end{center}
\end{titlepage}
\newpage

\tableofcontents
\newpage

\section{Theory}
\subsection{Exercise 1}
Given the following table:
\vspace{5pt}
\begin{equation*} 
    Q(s,a) \:=\:
    \begin{pmatrix}
        Q(1,1) & Q(1,2)\\
        Q(2,1) & Q(2,2)
    \end{pmatrix}
    \:=\: 
    \begin{pmatrix}
        1 & 2\\
        3 & 4
    \end{pmatrix}
\end{equation*}
\vspace{5pt}

\noindent Assuming $\alpha \:=\: 0.1$ and $\gamma \:=\: 0.5$, after the experience: $(s, a, r, s^{\prime}) = (1,2,3,2)$ we can compute the Q-table update for:
\begin{enumerate}
    \item Q-Learning 
    \item SARSA in the case $a' \:=\: \pi_{\epsilon}(s^{\prime}) \:=\: 2$
\end{enumerate} 
1. In the case of Q-Learning we can proceed using the following update rule:
\begin{equation*}
    Q(s,a) = Q(s,a) \:+\: \alpha [ \:r \:+\: \gamma(max_{a}Q(s^{\prime},a')) \:-\: Q(s,a)]
\end{equation*}
So we would have:
\begin{align*}
    && Q(s,a) = Q(1,2) \:+\: 0.1 [ \:3 \:+\: 0.5(max_{a}Q(2,a')) \:-\: Q(1,2)] \\
    && \\
    &\rightarrow& Q(1,2) \:+\: 0.1 [ \:3 \:+\: 0.5(max(Q(2,1), \:Q(2,2))) \:-\: Q(1,2)] \\
    && \\
    &\rightarrow&2 \:+\: 0.1 [ \:3 \:+\: 0.5\cdot4 \:-\: 2] \\
    && \\
    &\rightarrow&2 \:+\: 0.3 = 2.3 \\
\end{align*}
\vspace{5pt}

\noindent 2. For SARSA we use as update rule:
\begin{equation*}
    Q(s,a) = Q(s,a) \:+\: \alpha [ \:r \:+\: \gamma(Q(s^{\prime},a')) \:-\: Q(s,a)]
\end{equation*}
in this case $a'=2$, so we would have:
\begin{align*}
    && Q(s,a) = Q(1,2) \:+\: 0.1 [ \:3 \:+\: 0.5(Q(2,a'))\:-\: Q(1,2)] \\
    && \\
    &\rightarrow& Q(1,2) \:+\: 0.1 [ \:3 \:+\: 0.5(Q(2,2))\:-\: Q(1,2)] \\
    && \\
    &\rightarrow&2 \:+\: 0.1 [ \:3 \:+\: 0.5\cdot4 \:-\: 2) \\
    && \\
    &\rightarrow&2 \:+\: 0.3 = 2.3 \\
\end{align*}
\newpage


\subsection{Exercise 2}
The goal of this exercise is to prove that the n-step error can also be written as a sum of TD errors if the value estimates do not
change from step to step. In other words we have to show that the following equation is true:
\begin{equation}
    G_{t:t+n} \:-\: V_{t+n-1}(S_t) \:=\: \sum_{k=t}^{t+n-1} \gamma^{k-t} \delta_{k}
\end{equation}
\vspace{5pt}

\noindent In order to proceed with the demonstration we can expand the left side of the equation by taking in account that: 
\begin{equation}
    G_{t:t+n}  \:=\: R_{t+1} \:+\: \gamma R_{t+2} \:+\: ... \:+\: \gamma^{n-1}R_{t+n} \:+\: \gamma^{n}V_{t+n-1}(S_{t+n}) 
\end{equation}
\vspace{5pt}
\noindent After making the expansion we obtain:
% \setcounter{equation}{0}
\begin{flalign*}
    &G_{t:t+n} \:-\: V_{t+n-1}(S_t) \:=\:  R_{t+1} \:+\: \gamma R_{t+2} \:+\: ... \:+\: \gamma^{n-1}R_{t+n} \:+\: \gamma^{n}V_{t+n-1}(S_{t+n}) \:-\: V_{t+n-1}(S_t)&\\
    &&\\
    &\:=\:  R_{t+1} \:+\: \gamma R_{t+2} \:+\: ... \:+\: \gamma^{n-1}R_{t+n} \:+\: \gamma^{n}V_{t+n-1}(S_{t+n}) \:-\: V_{t+n-1}(S_t)& 1.&\\
    &&\\
    &\:=\:  R_{t+1} \:+\: \gamma R_{t+2} \:+\: ... \:+\: \gamma^{n-1}R_{t+n} \:+\: \gamma^{n}V(S_{t+n}) \:-\: V(S_t) & 2.&\\
    &&\\
    &\:=\:  \delta_t \:-\: \gamma V(S_{t+1}) \:+\: V(S_t) \:+\: \gamma R_{t+2} \:+\: ... \:+\: \gamma^{n-1}R_{t+n} \:+\: \gamma^{n}V(S_{t+n}) \:-\: V(S_t) & 3.& \\
    &&\\
    &\:=\: [\delta_t \:-\: \gamma V(S_{t+1}) \:+\: V(S_t)] \:+\: \gamma[\delta_{t+1} \:-\: \gamma V(S_{t+2}) \:+\: V(S_{t+1})]  \:+\: ...&\\
    &... \:+\: \gamma^{n-1} [\delta_{t+n-1} \:-\: \gamma V(S_{t+n}) \:+\: V(S_{t+n-1})] \:+\: \gamma^{n}V(S_{t+n}) \:-\: V(S_t)& 4.& \\
    &&\\
    &\sum_{k=t}^{t+n-1} [\gamma^{k-t}\delta_k \:-\: \gamma^{k-t+1} V(S_{k+1}) \:+\: \gamma^{k-t}V(S_k)] \:+\: \gamma^{n}V(S_{t+n}) \:-\: V(S_t) & 5.& \\
    &&\\
    &\sum_{k=t}^{t+n-1} [\gamma^{k-t}\delta_k] \:-\: \sum_{k=t}^{t+n-1} [\gamma^{k-t+1} V(S_{k+1})] \:+\: \gamma^{n}V(S_{t+n}) \:+\: \sum_{k=t}^{t+n-1} [\gamma^{k-t}V(S_k)] \:-\: V(S_t) & 6.&
\end{flalign*}
\vspace{5pt}
\begin{itemize}
    \item[2.] Substituting $V_t$ with $V$, under the assumpion that it does not change.
    \item[3.] Recalling that $\delta_t  \:=\: R_{t+1} \:+\: \gamma V(S_{t+n}) \:-\: V(S_t)$.
    \item[4.] Iterating the process of the previous steps for all the rewards $R_t$.
    \item[5.] Collecting the similar terms into a single summation.
    \item[6.] Splitting the summation in three parts, this will be useful to verify the presence of similar terms with opposite sign.
\end{itemize}
\newpage

\noindent We can rewrite the second summation at step 6. as: 
\begin{equation*}
    \:-\sum_{k=t}^{t+n-1} [\gamma^{k-t+1} V(S_{k+1})] \:+\: \gamma^{n}V(S_{t+n}) \:=\: \:-\sum_{k=t}^{t+n-2} [\gamma^{k-t+1} V(S_{k+1})] \:=\: \:-\sum_{k=t+1}^{t+n-1} [\gamma^{k-t}V(S_k)]
\end{equation*}
\vspace{5pt}
And the third summation at the same step as well:
\begin{equation*}
    \sum_{k=t}^{t+n-1} [\gamma^{k-t}V(S_k)] \:-\: V(S_t) \:=\: \sum_{k=t+1}^{t+n-1} [\gamma^{k-t}V(S_k)]
\end{equation*}
\vspace{5pt}
In this way our final equation becomes:
\begin{equation*}
    G_{t:t+n} \:-\: V_{t+n-1}(S_t) \:=\: \sum_{k=t}^{t+n-1} [\gamma^{k-t}\delta_k] \:-\sum_{k=t+1}^{t+n-1} [\gamma^{k-t}V(S_k)] \:+\: \sum_{k=t+1}^{t+n-1} [\gamma^{k-t}V(S_k)]
\end{equation*}
\vspace{5pt}
Which simplify in:
\begin{equation*}
    G_{t:t+n} \:-\: V_{t+n-1}(S_t) \:=\: \sum_{k=t}^{t+n-1} \gamma^{k-t}\delta_k
\end{equation*}

\newpage
\section{Code Implementation}
\subsection{RBF Q-Learning}

\newpage

\subsection{SARSA-Lambda}

\end{document}
